
\ctusetup{
	preprint = \ctuverlog,
%	mainlanguage = english,
    titlelanguage = czech,
	mainlanguage = czech,
	otherlanguages = {slovak,english},
	title-czech = {Detekce objektů z hloubkové kamery},
	title-english = {Object Detection from Depth camera},
	%subtitle-czech = {Cesta do tajů kdovíčeho},
	subtitle-english = {Journey to the who-knows-what wondeland},
	doctype = B,
	faculty = F3,
	department-czech = {Katedra kybernetiky},
	department-english = {Department of Cybernetics},
	author = {Lukáš Kunt},
	supervisor = {RNDr. Petr Štěpán, Ph.D.},
	supervisor-address = {Ústav X, \\ Uliční 5, \\ Praha 99},
	%supervisor-specialist = {John Doe},
	fieldofstudy-english = {Mathematical Engineering},
	subfieldofstudy-english = {Mathematical Modelling},
	%fieldofstudy-czech = {Matematcké inženýrství},
	subfieldofstudy-czech = {Kybernetika a robotika},
	keywords-czech = {slovo, klíč},
	keywords-english = {word, key},
	day = 11,
	month = 5,
	year = 2020,
	%specification-file = {ctutest-zadani.pdf},
%	front-specification = true,
%	front-list-of-figures = false,
%	front-list-of-tables = false,
%	monochrome = true,
%	layout-short = true,
}

\ctuprocess

\addto\ctucaptionsczech{%
	\def\supervisorname{Vedoucí}%
	\def\subfieldofstudyname{Studijní program}%
}

\ctutemplateset{maketitle twocolumn default}{
	\begin{twocolumnfrontmatterpage}
		\ctutemplate{twocolumn.thanks}
		\ctutemplate{twocolumn.declaration}
		\ctutemplate{twocolumn.abstract.in.titlelanguage}
		\ctutemplate{twocolumn.abstract.in.secondlanguage}
		\ctutemplate{twocolumn.tableofcontents}
		\ctutemplate{twocolumn.listoffigures}
	\end{twocolumnfrontmatterpage}
}

% Theorem declarations, this is the reasonable default, anybody can do what they wish.
% If you prefer theorems in italics rather than slanted, use \theoremstyle{plainit}
%\theoremstyle{plain}
%\newtheorem{theorem}{Theorem}[chapter]
%\newtheorem{corollary}[theorem]{Corollary}
%\newtheorem{lemma}[theorem]{Lemma}
%\newtheorem{proposition}[theorem]{Proposition}

%\theoremstyle{definition}
%\newtheorem{definition}[theorem]{Definition}
%\newtheorem{example}[theorem]{Example}
%\newtheorem{conjecture}[theorem]{Conjecture}

%\theoremstyle{note}
%\newtheorem*{remark*}{Remark}
%\newtheorem{remark}[theorem]{Remark}

%\setlength{\parskip}{5ex plus 0.2ex minus 0.2ex}
